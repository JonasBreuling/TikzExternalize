%\documentclass{article}
\documentclass[a4paper]{article}

% enable blind text
\usepackage{lipsum}

%%%%%%%%%%%%%%%%%%%%%%%%%%%%%%%%%%%%%%%%%%%%%%%%%%%%%%%%%%%%%%%%%%%%%%%%%%%%%%%%%%%%%%%%%%
% begin interesting code
%%%%%%%%%%%%%%%%%%%%%%%%%%%%%%%%%%%%%%%%%%%%%%%%%%%%%%%%%%%%%%%%%%%%%%%%%%%%%%%%%%%%%%%%%%

% include tikz and calc library for nice image
\usepackage{tikz}
\usetikzlibrary{calc}

% for TeXstudio see  here:https://tex.stackexchange.com/questions/170291/tikz-wont-let-me-externalize-and-halts-on-error
% you need to make sure that you include --shell-escape parameter when invoking pdflatex in TeXstudio:
%	- goto: Options > Configure TeXStudio > Commands
% 	- change from: pdflatex -synctex=1 -interaction=nonstopmode %.tex
%            to:   pdflatex -synctex=1 -interaction=nonstopmode --shell-escape %.tex 
\usetikzlibrary{external}

% use this to get additionally .eps files using tikzexternalize
\tikzset{external/system call={pdflatex \tikzexternalcheckshellescape 
		-halt-on-error
		-interaction=batchmode 
		-jobname "\image" "\texsource"
		&& pdftops -eps "\image.pdf"}}

% enable externalize: write tikzpicture in separate .pdf file
\tikzexternalize[prefix=figures/, shell escape=-enable-write18]

%%%%%%%%%%%%%%%%%%%%%%%%%%%%%%%%%%%%%%%%%%%%%%%%%%%%%%%%%%%%%%%%%%%%%%%%%%%%%%%%%%%%%%%%%%
% end interesting code
%%%%%%%%%%%%%%%%%%%%%%%%%%%%%%%%%%%%%%%%%%%%%%%%%%%%%%%%%%%%%%%%%%%%%%%%%%%%%%%%%%%%%%%%%%

% enables usage of hyperlinks and hide ugly borders
\usepackage[hidelinks]{hyperref}

% image setup I
\tikzset{3D/.cd,
	x/.store in=\xx, x=0,
	y/.store in=\yy, y=0,
	z/.store in=\zz, z=0
}

% image setup II
\tikzdeclarecoordinatesystem{3D}{%
	\tikzset{3D/.cd,#1}%
	\pgfpoint{sin(\yy)*(\xx)}{-((\xx)/75)^2+(\zz)/100*(\xx)}%
}

\begin{document}
	
	% definitions for titlepage
	\title{Tikz Externalize Usage}
	\date{\today}
	\author{Jonas Harsch \\ University Stuttgart}
	
	% make a titlepage
	\clearpage\maketitle
	\thispagestyle{empty}
	
	% go to next page
	\newpage
	
	% make a section
	\section{Lorem ipsum}
	
	% some blindtext
	\lipsum[1]
	
	% nice tikz figure centered in a LaTeX figure
	\begin{figure}[h!]
	\centering
	
	% tikzpicture from http://www.texample.net/tikz/examples/dominoes/
	\begin{tikzpicture}[line join=round, very thin, scale=0.6]
	\def\e{1260}
	\foreach \x [evaluate={\i=mod(\x+90,360); \j=int((\i<180)*2-1); \t=3;
		\sc=\x/\e; \n=int((\e-\x)/15+5); \X=\x/\e;}] in {10,25,...,\e} {
		
		\path [shift={(3D cs:x=\x-\t,y={3*sin(\x-\t)})}, yslant=cos(\x)/5]
		(-\X/2, 0)   coordinate (A')  ( \X/2, 0)   coordinate (B')
		( \X/2,2*\X) coordinate (C')  (-\X/2,2*\X) coordinate (D');
		
		\path [shift={(3D cs:x=\x,y=3*sin \x)}, yslant=cos(\x)/5]
		(-\X/2, 0)   coordinate (A) ( \X/2, 0)   coordinate (B)
		( \X/2,2*\X) coordinate (C) (-\X/2,2*\X) coordinate (D);
		
		\filldraw [black!90] (B) -- (B') -- (C') -- (C)  -- cycle;
		\filldraw [black!80] (A) -- (A') -- (D') -- (D)  -- cycle;
		\filldraw [black!70] (C) -- (D)  -- (D') -- (C') -- cycle;
		\filldraw [black]    (A) -- (B)  -- (C)  -- (D)  -- cycle;
		
		\node [text=white, shift={($(C)!0.5!(D)$)}, anchor=north,
		yslant=cos(\x)/5, font=\sf, scale=\sc*1.5] at (0,-.33*\X) {\n};
	}
	%
	\foreach \i [evaluate={\x=\i*30-10; \X=1; \n=int(5-\i); \xsl=\x/180}]
	in {1,...,4} {
		
		\path [shift={(3D cs:x=\x+\e,y=-3*\x/90)}, yslant=cos \e/5, xslant=\xsl]
		(-\X/2, 0)           coordinate (A) ( \X/2, 0)           coordinate (B)
		( \X/2, \X*2-\x/360) coordinate (C) (-\X/2, \X*2-\x/360) coordinate (D);
		
		\path [shift={(3D cs:x=\x+\e,y=-3*\x/90)}, shift={(5/50,5/50-\i*2/50)},
		yslant=cos \e/5, xslant=\xsl]
		(-\X/2, 0)           coordinate (A') ( \X/2, 0)           coordinate (B')
		( \X/2, \X*2-\x/330) coordinate (C') (-\X/2, \X*2-\x/330) coordinate (D');
		
		\filldraw [black!70] (C) -- (D)  -- (D') -- (C') -- cycle;
		\filldraw [black!70] (A) -- (B)  -- (B') -- (A') -- cycle;
		\filldraw [black!90] (B) -- (B') -- (C') -- (C)  -- cycle;
		\filldraw [black]    (A) -- (B)  -- (C)  -- (D)  -- cycle;
		
		\node [text=white, shift={($(C)!0.5!(D)$)}, anchor=north, xslant=\xsl,
		yslant=cos \e/5, font=\sf, scale=1.5] at (0,-.33*\X) {\n};
	}
	\end{tikzpicture}
	\caption{Falling dominoes, see \url{http://www.texample.net/tikz/examples/dominoes/}.}
	\end{figure}
	
	% again some blindtext
	\lipsum[2]
	
\end{document}